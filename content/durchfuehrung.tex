\section{Durchführung}
\label{sec:Durchführung}
\subsection{Entladevorgang des RC-Kreises und Bestimmung der Zeitkonstanten}
(SCHALTBILD EINFÜGEN)
Um den Entladevorgang eines RC-Kreises zu untersuchen,
wird die Spannung am Kondensator mit Hilfe eines
Oszilloskopes beobachtet. An das RC-Glied wird ein 
Generator angeschlossen, welcher eine Rechteckspannung
mit einer Amplitude von $U_0=(WERT)$ erzeugt.
Sobald der Kondensator voll aufgeladen ist und die Spannung
auf null springt, beginnt die Entladung des Kondensators.
Der Vorgang endet, wenn die Spannung wieder $U=U_0=Wert$
beträgt. Der Entladevorgang kann also nicht vollständig
aufgezeichnet werden, das Oszilloskop zeigt jedoch
einen entsprechenden Ausschnitt (REFERENZ BILD) mit dem
sich die Zeitkonstante RC bestimmen lässt.


\subsection{Frequenzabhängigkeit der Phasenverschiebung und der Amplitude beim Kondensator}
(BILD VOM AUFBAU EINFÜGEN)
Der Aufbau entspricht dem von (REFERENZ) wobei hier
zusätzlich die Spannung $U$ des Generators auf dem
zweiten Kanal des Oszillographen angezeigt wird und
somit mit der Spannung $U_C$ des Kondensators verglichen
werden kann. Außerdem liegt nun keine Rechteckspannung,
sondern eine Sinusspannung mit $U_0=WERT$ vor.
Die Phasenverschiebung $\phi$ zwischen $U$ und $U_C$
und die Amplitude von $U_C$ werden nun gleichzeitig
untersucht. Dafür wird die Frequenz $\omega$ der
Generatorspannung $U$ von 10\,Hz bis 10\,000\,Hz
hochgeregelt und 21 Messwerte ins Messheft aufgenommen.

\subsection{Integrierfunktion des RC-Kreises}
Es wird der gleiche Aufbau wie bei (REFERENZ) verwendet.
Nach der Bedingung $\omega \gg 1$/$RC$ wird
eine Frequenz von $\omega=WERT$ gewählt.
Anschließend soll die zu integrierende und integrierte
Spannung auf dem Zweikanaloszilloskop bei einer
Rechteck-, Sinus- und Dreieckspannung dargestellt werden.
Die Amplitude der Generatorspannung $U$ ist hierbei
(WERT).
