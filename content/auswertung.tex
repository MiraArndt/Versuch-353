\section{Auswertung}
\label{sec:Auswertung}

\subsection{Bestimmung der Zeitkonstante eines RC-Gliedes}

Durch die Beobachtung des Spannungsverlaufs vom Auf- beziehungsweise Entladevorgang des 
Kondensators wurden Paare von Messwerten $(U_(t),t)$ aufgenommen. Diese wurden in 
Tabelle \ref{tab:1} aufgeführt und anschließend in einem halblogarithmischen Diagramm
aufgetragen. Die Generatorspannung betrug
konstant $U_0 = 9.8\,/,\ \si{\volt}$. Mithilfe einer linearen Ausgleichsrechnung,
welche die Gleichung 
\begin{equation}
np.log(\frac{U_C(t)}{U_C(t)}) = -\frac{1}{RC} \cdot t + a
\end{equation}
\noindent
als Ansatz verwendet, der aus Gleichung \ref{eq:Auswertung1} folgt, ergeben sich für die 
Koeffizienten $RC$ und $a_1$ folgende Werte
\begin{align}
RC  =  (1.343 \pm 0.01809) \cdot 10^-3\, {\si{\second}}      \nonumber \\
a =  (0.007 \pm 0.02570) \cdot 10^-3.                      \nonumber
\end{align}
\noindent
Der Wert für RC lässt sich mit dem tatsächenlichen Wert für die Zeitkonstante vergleichen. 
Die dazu benötigten Werte $R = 15058 \pm 0.6\, \si{\ohm}$ und $C = 93.2\, \si{\nano\farad}$ 
wurden vorgegeben. Somit ergibt sich als Vergleichswert 
\begin{align}
RC = (15058 \pm 0.6)\, \si{\ohm} \cdot 93.2\, \si{\nano\farad} \nonumber \\
\iff RC = (1.40341 \pm 0.00006)\, \si{\second}.  \nonumber \\
\end{align}














\subsection{Zweiter Weg zur Berechnung der Zeitkonstante RC}
Die zu dem Teil des Versuchs gehörenden Messwerte sind in der Tabelle (Referenz) zu finden.
Mithilfe einer nichtlinearen Regression soll die Zeitkonstante RC ermittelt werden. Hierzu
ist der Ansatz
\begin{align}
    \frac{A(\nu_i)}{U_0} = np.exp(-a\nu_i+b)
\end{align}
\noindent
zu verwenden, wobei 


\subsection{Der RC-Kreis als Integrator}

Wenn unterschiedliche Spannungen bei hohen Frequenzen auf den RC-Kreis gegeben werden, 
lässt sich erkennen, dass dieser als Integrator arbeiten kann. Wird eine Rechteckspannung
auf den RC-Kreis gegeben, so gibt dieser eine Spannung mit konstanter Steigung aus 
(siehe Abbildung). Falls eine Dreiecksspannung auf den RC-Kreis gegeben wird, lässt sich eine
parabelförmige Spannung ausmachen (vergleiche dazu Abbildung) und wird eine Sinusspannung auf den 
RC-Kreis gegeben, so gibt dieser eine Cosinusspannung aus (siehe Abbildung 3).

